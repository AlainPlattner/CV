\documentclass[10pt]{article}

%\usepackage[latin1]{inputenc}
%\usepackage[english]{babel}
\usepackage{graphicx}
\usepackage{color}
\usepackage{amsmath}
\usepackage{amssymb}
\usepackage{hyperref}
\usepackage{pbox}

\setlength{\parindent}{0cm}

\oddsidemargin -0in%5mm
\evensidemargin -0in%5mm
\topmargin -0.7in%-0.8in%-10mm
%\textheight 9.5in%9.5in%230mm
\textheight 9.2in
\textwidth 6.4in%150mm

%\pagestyle{empty} 

\begin{document}

%\hyphenpenalty=10000
%\sloppy

\hyphenation{}


\newcommand{\lkon}{\begin{color}{blue}}
\newcommand{\lkoff}{\end{color}}

\newcommand{\spc}{\vspace{0.5cm}}
\newcommand{\spcp}{\vspace{0.2cm}}

\newcommand{\tsize}{\large}

\vspace*{-0.3in}
\makebox[1.072\linewidth][c]
{
%\begin{figure}
  %\vspace{-0.3in}
\hspace{-1cm}
\begin{minipage}{0.3\columnwidth}
\flushleft
\includegraphics[height=0.8in,trim = 7.295cm 3.53cm 1.24cm 1.18cm, clip]%0.9in]
{CSUF_logo}
\end{minipage}
\hfill
\begin{minipage}{0.5\columnwidth}
  %\flushright {\LARGE Alain Plattner}\\
  \centering {\LARGE Alain Plattner}\\
  \lkon\url{aplattner@csufresno.edu}\lkoff\\
  \lkon\url{http://alainplattner.net}\lkoff\\
\end{minipage}
%\hfill
%\hspace*{1cm}
\begin{minipage}{0.3\columnwidth}
  \flushright
  \includegraphics[height=0.8in,trim = 2.85cm 2cm 2.85cm 0.5cm,clip]{FSGG_logo}
\end{minipage}
%\end{figure}
}



\vspace*{1cm}
\textbf{\tsize CV date:} {\tsize \today}

\spcp
For the most recent version of my \href{http://alainplattner.net/downloads/Plattner_CV.pdf}{CV} click \href{http://alainplattner.net/downloads/Plattner_CV.pdf}{\lkon\underline{here}\lkoff}

\spc
\textbf{\tsize Expertise:}

\spcp
Planetary crustal magnetic fields. Regional
  inversion of satellite magnetic data. Regional spherical-harmonic spectral
  analysis.  Electrical resistivity tomography. Near-surface geophysics.  

\spc
\textbf{\tsize Address:}

\spcp
Department of Earth and Environmental Sciences\\
California State University, Fresno\\
2576 E. San Ramon Ave., Mail Stop ST-24\\
Fresno, CA 93740


\spc
\textbf{\tsize Positions:}

\spcp
2014--present:
Assistant Professor at Department of Earth and Environmental Sciences, \\California State University Fresno, Fresno CA

\spcp
2011--2014:
Postdoctoral Researcher at the Department of Geosciences, Princeton University, \\Princeton NJ
%Research topic: Development of vectorial Slepian functions and their application to the estimation of crustal magnetization from satellite data.


\spc
\textbf{\tsize Degrees:}

\spcp
\par{2011:} PhD (Dr. Sc. ETH Zurich) in Geophysics at the Institute of Geophysics,
ETH Zurich, Switzerland.
Thesis title: \emph{ Adaptive wavelet methods for geoelectric modelling and inversion}.
Adviser: Prof. Hansruedi Maurer.
doi: 10.3929/ethz-a-006481159

\spcp
\par{2006:} Master of Science in Mathematics at the Institute of Mathematics,
 University of Basel, Switzerland.
 Majors: Algebraic geometry, numerical mathematics

 \spcp
\par{ 2004:} Bachelor of Science in Mathematics at the Institute of Mathematics, 
University of Basel, Switzerland.



\spc
\textbf{\tsize Publications:}

\spcp
$^*$: Undergraduate student first author

\spcp
\emph{Research articles/book chapters:}

\spcp
\hspace{-0.835cm}$^*$[10] A.~R.~Robbins, \textbf{A.~Plattner} (2018),
Offset-electrode profile acquisition strategy for
electrical resistivity tomography,
\emph{J.~Appl.~Geoph.}, 151:66--72, \href{https://www.sciencedirect.com/science/article/pii/S0926985117308376?via%3Dihub}{doi:~10.1016/j.jappgeo.2018.01.027} 


\spcp
\hspace{-0.5cm}[9] \textbf{A.~Plattner}, F.~J.~Simons (2017),
Internal and external potential field estimation
from regional gradient data at varying satellite altitude,
\emph{Geophys.~J.~Int.}, 211(1):207--238, \href{https://academic.oup.com/gji/article-lookup/doi/10.1093/gji/ggx244}{doi:~10.1093/gji/ggx244} 

\spcp
\hspace{-0.5cm}[8] \textbf{A.~Plattner}, F.~J.~Simons (2015),
High-resolution local magnetic field models for the
Martian South Pole
from Mars Global Surveyor data,
\emph{J.~Geophys.~Res.}, 120:1543--1566,
\href{http://onlinelibrary.wiley.com/doi/10.1002/2015JE004869/abstract}{doi: 10.1002/2015JE004869}

\spcp
\hspace{-0.5cm}[7] C.~Harig, K.~W.~Lewis, \textbf{A.~Plattner}, and F.~J.~Simons (2015),
A suite of software analyzes data on the sphere,
\emph{Eos Trans.~AGU}, 96(6):18--22,
\href{https://eos.org/project-updates/a-suite-of-software-analyzes-data-on-the-sphere-2}{doi: 10.1029/2015EO025851}

\spcp
\hspace{-0.5cm}[6] \textbf{A.~Plattner} and F.~J.~Simons (2015),
Potential field estimation using scalar and vector Slepian functions at satellite altitude,
\emph{Handbook of Geomathematics, 2nd edition},
\href{https://link.springer.com/referenceworkentry/10.1007\%2F978-3-642-27793-1_64-2}{doi: 10.1007/978-3-642-27793-1\_64-2}

\spcp
\hspace{-0.5cm}[5] F.~J.~Simons and \textbf{A.~Plattner} (2015),
Scalar and vector Slepian functions, spherical signal estimation and spectral analysis,
\emph{Handbook of Geomathematics, 2nd edition},
\href{https://link.springer.com/referenceworkentry/10.1007\%2F978-3-642-27793-1_30-2}{doi: 10.1007/978-3-642-27793-1\_30-2}

\spcp
\hspace{-0.5cm}[4] \textbf{A.~Plattner} and F.~J.~Simons (2014),
Spatiospectral concentration of vector fields on a sphere,
\emph{Appl.~Comput.~Harmon.~Anal.}, 36(1):1--22, 
\href{http://www.sciencedirect.com/science/article/pii/S106352031300002X?via\%3Dihub}{doi: 10.1016/j.acha.2012.12.001}

\spcp
\hspace{-0.5cm}[3] \textbf{A.~Plattner} and F.~J.~Simons (2013), 
A spatiospectral localization approach for analyzing and repres enting vector-valued functions on spherical surfaces,
\emph{Proc. SPIE} 8858, Wavelets and Sparsity XV, 88580N,
\href{http://proceedings.spiedigitallibrary.org/proceeding.aspx?articleid=1745029}{doi: 10.1117/12.2024703}

\spcp
\hspace{-0.5cm}[2] \textbf{A.~Plattner}, H.~R.~Maurer, J.~Vorloeper and M.~Blome (2012),
3-D electrical resistivity tomography using adaptive wavelet parameter grids,
\emph{Geophys.~J.~Int.}, 189(1):317--330,
\href{https://academic.oup.com/gji/article-lookup/doi/10.1111/j.1365-246X.2012.05374.x}{doi: 10.1111/j.1365-246X.2012.05374.x}

\spcp
\hspace{-0.5cm}[1] \textbf{A.~Plattner}, H.~R.~Maurer, J.~Vorloeper and W.~Dahmen (2010),
Three-dimensional geoelectric modelling with optimal work/accuracy rate 
using an adaptive wavelet algorithm,
\emph{Geophys.~J.~Int.}, 182(2):741--752,
\href{https://academic.oup.com/gji/article-lookup/doi/10.1111/j.1365-246X.2010.04677.x}{doi: 10.1111/j.1365-246X.2010.04677.x}

\spc
\emph{Extended abstracts:}

\spcp
\hspace{-0.5cm}[4] \textbf{A.~Plattner}, G.~J.~Golabek, F.~J.~Simons (2017),
A spectral view of the Terra Sirenum / Cimmeria crustal magnetic
field,
\emph{48th Lunar and Planetary Science Conference 2017},
\href{http://www.lpi.usra.edu/meetings/lpsc2017/pdf/1627.pdf}{Abstract 1627}

\spcp
\hspace{-0.67cm}$^*$[3] A.~R.~Robbins and \textbf{A.~Plattner}
(2017),
2.75-D ERT: Zigzag electrode acquisition strategy to improve 2-D
Profiles,
\emph{Symposium on the Application of Geophysics to Engineering and
  Environmental Problems 2017}, 183--187,
\href{http://library.seg.org/doi/pdf/10.4133/SAGEEP.30-007}{doi: 10.4133/SAGEEP.30-007}

\spcp
\hspace{-0.5cm}[2] \textbf{A.~Plattner} and F.~J. Simons (2015),
Mars' heterogeneous South Polar magnetic field revealed using altitude vector Slepian functions,
\emph{46th Lunar and Planetary Science Conference 2015},
\href{http://www.hou.usra.edu/meetings/lpsc2015/pdf/1794.pdf}{Abstract 1794}

\spcp
\hspace{-0.5cm}[1] \textbf{A.~Plattner}, F.~J.~Simons, L.~Wei (2012),
Analysis of real vector fields on the sphere using Slepian functions,
\emph{IEEE Statistical Signal Processing Workshop (SSP)},
\href{http://ieeexplore.ieee.org/stamp/stamp.jsp?tp=&arnumber=6319659}{Abstract}


\spc
\textbf{\tsize Talks:}

\spcp
$^*$: Undergraduate student first author

\spcp
\emph{Invited/solicited conference talks:} 

\spcp 
\hspace{-0.4cm} $^*$ 2.75-D ERT: Zigzag electrode acquisition strategy
to improve 2-D profiles,
A.~R.~Robbins, A.~Plattner,
\emph{23rd European Meeting of Environmental and Engineering Geophysics}, Malmo, Sweden, Sep 2017 (best of SAGEEP)

\spcp
High-resolution crustal magnetic field model of the Martian South Pole using altitude vector\\ Slepian functions,
A.~Plattner, F.~J.~Simons,
\emph{Joint Mathematics Meeting}, San Antonio, TX, Jan 2015 (invited)

\spcp
Planetary potential-field inversion from vectorial data: Using Slepian functions for varying satellite altitude,
A.~Plattner, F.~J.~Simons,
\emph{Joint Mathematics Meeting}, Baltimore, MD, Jan 2014 (invited)

\spcp
Regional crustal field modeling from regional satellite data with varying altitude using dedicated vector Slepian functions,
A.~Plattner, F.~J.~Simons,
\emph{AGU Fall Meeting}, San Francisco, CA, Dec 2013 (invited)

\spcp
Signal and Spectral Estimation on a Sphere,
F.~J.~Simons, A.~Plattner,
\emph{AMMCS 2013}, Waterloo, ON, Canada, August 2013 (invited speaker: F.J.~Simons)

\spcp
Vectorial Slepian functions and the estimation of the crustal magnetic field,
A.~Plattner, F.~J.~Simons,
\emph{EGU General Assembly}, Vienna, Austria, April 2013 (solicited)

\spc
\emph{Regular conference talks:}

\spcp
A spectral view of the Terra Sirenum / Cimmeria crustal magnetic field, A.~Plattner, F.J.~Simons, G.~Golabek, \emph{48th Lunar and Planetary Science Conference}, Houston, TX, March 2017

\spcp 
\hspace{-0.4cm} $^*$ 2.75-D ERT: Zigzag electrode acquisition strategy
to improve 2-D profiles,
A.~R.~Robbins, A.~Plattner,
\emph{SAGEEP}, Denver, CO, Mar 2017

\spcp
The Crustal Magnetic Field of Terra Sirenum and Cimmeria, Mars. A Spectral Perspective,
A.~Plattner, F.~J.~Simons, G.~Golabek, 
\emph{AGU Fall Meeting}, San Francisco, CA, Dec 2016

\spcp
Teaching Near-Surface Geophysics within the Matlab/Octave Community,
A.~Plattner, 
\emph{AGU Fall Meeting}, San Francisco, CA, Dec 2016

\spcp
Localized Bandlimited Inversion of Planetary Magnetic-Field Data,
A.~Plattner, F.~J.~Simons,
\emph{SIAM Conference on Mathematical and Computational Issues in the Geosciences},
Stanford University, Stanford, CA, Jul 2015

\spcp
Source field estimation from satellite data using vectorial spatiospectrally 
concentrated functions,
A.~Plattner, F.~J.~Simons,
\emph{Geomathematics 2013}, Sankt Martin, Germany, April 2013

\spcp
Vector-valued crustal magnetic field estimation using vector Slepian functions,
A.~Plattner, F.~J.~Simons,
\emph{AGU Fall Meeting}, San Francisco, CA, Dec 2012

\spcp
Geophysical survey of the Peristeries plateau in Polis Chrysochous, Cyprus,
A.~Plattner, F.~J.~Simons, J.~S.~Smith, A.~C.~Maloof, J.~Husson,
\emph{American Schools of Oriental Research Annual Meeting}, Chicago, IL, Nov 2012

\spcp
Adaptive wavelet parameterization for 3d electrical resistivity tomography,
A.~Plattner, H.~R.~Maurer, 
\emph{AGU Fall Meeting}, San Francisco, CA, Dec 2011

\spcp
Adaptive wavelet modeling of geophysical data,
A.~Plattner, H.~R.~Maurer, J.~Vorloeper and W.~Dahmen, 
\emph{AGU Fall Meeting}, San Francisco, CA, Dec 2009


\spc
\emph{Webinar talks:}

\spcp
Examples using Matlab / Octave for Experimental Design and Data Processing in a Near-surface Applied Geophysics Class, 
\emph{Developing Computational Skills in the Sciences with Matlab}, April 2017\\
\url{http://serc.carleton.edu/details/files/115169.html}

\spc
\emph{Seminar talks:}\\
\emph{NASA Goddard Space Flight Center (USA), July 2018}\\
\emph{University of Siegen (Germany), Department of Mathematics, May 2018}\\
\emph{University of the Witwatersrand (South Africa), School of Geosciences}, July 2016\\
\emph{University of British Columbia (CA), Dept.~of Earth, Ocean and Atmospheric Sci.}, Feb 2016\\
\emph{UC Santa Cruz (USA), Earth and Planetary Sciences Department}, Oct 2014\\
\emph{Princeton University (USA), Department of Geosciences}, Sept  2011, Apr 2014\\
\emph{CSU Fresno (USA), Department of Earth and Environmental Science}, Feb 2014\\
\emph{Princeton University (USA), Program in Appl.~and Comp.~Mathematics}, Nov 2013\\
\emph{Rutgers (USA), Department of Earth and Environmental Sciences}, Feb 2012\\
\emph{Cornell University (USA), Department of Earth and Atmospheric Sciences}, Feb 2012\\
\emph{Universite de Lausanne (Switzerland), Institute de Geophysique}, Nov 2009, Jan 2012\\
\emph{ETH Zurich (Switzerland), Seminar for Applied Mathematics}, Dec 2010\\
\emph{ETH Zurich (Switzerland), Department of Earth Sciences}, Oct 2009


\spc
\textbf{\tsize Posters:}

\spcp
$^*$: Undergraduate student first author

\spcp
Mercury's Crustal Magnetic Field from MESSENGER Data,
 A.~Plattner, C.~L.~Johnson 
\emph{AGU Fall Meeting}, New Orleans, LA, Dec 2017 

\spcp
\hspace{-0.4cm} $^*$ A glimpse in the third dimension for electrical
resistivity profiles,
A.~R.~Robbins, A.~Plattner,
\emph{AGU Fall Meeting}, New Orleans, LA, Dec 2017 

\spcp
\hspace{-0.4cm} $^*$ Electrical Resistivity and Ground Penetrating Radar 
Investigation of Presence and Extent of Hardpan Soil Layers,
 S.~J.~Thao, A.~Plattner,
\emph{AGU Fall Meeting}, San Francisco, CA, Dec 2015

\spcp
Localized crustal magnetic field inversion from inner- and outer-source altitude vector Slepian functions,
A.~Plattner,  F.~J.~Simons,
\emph{AGU Fall Meeting}, San Francisco, CA, Dec 2015

\spcp
Mars' Heterogeneous South Polar Magnetic Field Revealed using Altitude Vector Slepian Functions,
A.~Plattner,  F.~J.~Simons,
\emph{46th Lunar and Planetary Science Conference}, Houston, TX, March 2015

\spcp
High-resolution Local Crustal Magnetic Field Modeling of the Martian South Pole,
A.~Plattner,  F.~J.~Simons,
\emph{AGU Fall Meeting}, San Francisco, CA, Dec 2014

\spcp
Altitude vector Slepian functions and satellite crustal magnetic field data,
A.~Plattner,  F.~J.~Simons,
\emph{3rd Sward Science Meeting}, Copenhagen, Denmark, June 2014

\spcp
Local gravity field modeling from vectorial satellite data using Slepian functions,
A.~Plattner,  F.~J.~Simons,
\emph{AGU Fall Meeting}, San Francisco, CA, Dec 2013

\spcp
Analysis of real vector fields on the sphere using Slepian functions,
A.~Plattner, F.~J.~Simons, L.~Wei,
\emph{IEEE Statistical Signal Processing Workshop}, Ann Arbor, MI, Aug 2012

\spcp
Lithospheric magnetic field reconstruction using vector Slepian functions,
A.~Plattner, F.~J.~Simons,
\emph{Symposium on Study of the Earth's Deep Interior}, Leeds, United Kingdom, July 2012

\spcp
Spatiospectral concentration of vector fields on a sphere,
A.~Plattner, F.~J.~Simons,
\emph{Challenges in Geometry, Analysis, and Computation: High-Dimensional Synthesis}, 
New Haven, CT, June 2012

\spcp
Vector spherical Slepian functions -- spatiospectral concentration of vector fields on the sphere,
A.~Plattner, F.~J.~Simons, L.~Wei,
\emph{AGU Fall Meeting}, San Francisco, CA, Dec 2011




\spc
\textbf{\tsize Teaching:}

\spcp
\textbf{2017}

``Applied Geophysics'', Department of Earth and Environmental Science, CSU Fresno.

\spcp
``Natural Disasters and Earth Resources'', Department of Earth and Environmental Science, CSU Fresno.

\spcp
\textbf{2016}

``Geoscientific Computing'', Department of Earth and Environmental Science, CSU Fresno.

\spcp
``Natural Disasters and Earth Resources'', Department of Earth and Environmental Science, CSU Fresno.

\spcp
\textbf{2015}

``Near-surface geophysics'', Department of Earth and Environmental Science, CSU Fresno.

\spcp
``Natural Disasters and Earth Resources'', Department of Earth and Environmental Science, CSU Fresno.

\spcp
\textbf{2014}

``Geophysics Seminar'', Department of Earth and Environmental Science, CSU Fresno.

\spcp
``Environmental Earth and Life Science'', Department of Earth and Environmental Science, CSU Fresno.

\spcp
\textbf{2011--2013}

Instructor for Earth's environments and ancient civilizations'', Department of Geosciences,\\ Princeton University.

\spcp
\textbf{2006--2011}

Teaching assistant for Numerical modeling in applied geophysics'',  Institute of Geophysics, ETH Zurich.

\spcp        
Teaching assistant for field courses (electromagnetic prospecting),
Institute of Geophysics, ETH Zurich.

\spcp
\textbf{2004--2006}

Teaching assistant for ``Mathematics for natural scientists'', Institute of Mathematics, University of Basel.

\spcp
\textbf{2004}

Teaching assistant for ``Linear algebra'', Institute of Mathematics, University of Basel.

\clearpage
%\spc
\textbf{\tsize Funding:}

\spcp
  NSF Geoinformatics
  %\href{http://www.nsf.gov/awardsearch/showAward?\
  %  AWD_ID=1550732&HistoricalAwards=false}{\lkon[EAR-1550732]\lkoff},
  [EAR-1550732],
  2016--2019

 \spcp 
NASA Mars Data Analysis Program [NNX14AM29G], 2014--2017

\spcp
Swiss National Science Foundation Fellowship for Prospective Researchers
[PBEZP2-134427], 2011--2012

\spcp
Ulrich Schmucker Memorial Trust grant (2011)

\spc
\textbf{\tsize Advising:}

\spcp
Master's Students:\\
Marcus Pacheco, expected 2019 (California State University, Fresno)\\
Christine Liu, 2018 (California State University, Fresno)

\spcp
External PhD examiner:\\
Kathrin Seibert, 2018 (University of Siegen, Germany)\\
Timothy Wiese, 2012 (University of Adelaide, Australia)



\spc
\textbf{\tsize Service:}

\spcp
Serving on the College of Science and Mathematics curriculum committee at\\
California State University, Fresno.

\spcp
Reviewed 21 papers since 2010 for the following journals:\\
\emph{Earth-Sci.~Rev.},
\emph{Geophys.~J.~Int.} (7),
\emph{Geophys.~Prospect.},
\emph{Geophysics} (3),
\emph{IEEE T.~Signal Proces.},\\
\emph{Intern.~J.~Geomath.},
\emph{Int.~J.~Speleol.},
\emph{J.~Geodesy},
\emph{J.~Geophys.~Res.},
\emph{Mech.~Res.~Commun.},\\
\emph{Pure~Appl.~Geophys.} (2),
\emph{J.~Appl.~Geophys.}


\spcp
Co-organized and chaired session
``GP34A: Planetary Magnetism and Gravity''
at AGU Fall Meeting 2017.
          
\spcp
Co-organized and chaired session
``GP009: Imaging the crust using magnetic, gravity and electromagnetic methods''
at AGU Fall Meeting 2016.

\spcp
Co-organized the minisymposium
``Forward and Inverse Problems in Geodesy, Geodynamics, and Geomagnetism''
at SIAM Conference on Mathematical and Computational Issues in the Geosciences, July 2015.

\spcp
Appeared on the radio show ``Science, a candle in the dark'' on KFCF´
\url{https://itunes.apple.com/us/podcast/science-candle-in-dark-podcast/id972796179}

\spcp
Presented at the local Caf\'e Scientifique
\url{https://valleycafesci.wordpress.com/}




%
%\cventry{Service:}
%{%Organizer of the brown-bag seminar of the Department of Geosciences, Princeton University.
%%
%Between Feb 2012 and June 2012 I organized the brown-bag seminar for the Department of Geosciences, Princeton University.
%}


\end{document}

